\documentclass[a4]{scrartcl}
\usepackage[margin=2.5cm]{geometry}
\usepackage[francais]{babel}
\usepackage[utf8]{inputenc}
\usepackage{amsmath}
\usepackage{graphicx}
\usepackage[colorinlistoftodos]{todonotes}
\usepackage{tabularx}
\usepackage{colortbl}
\usepackage{hyperref}


% define title
\title{Réalité Virtuelle}
\subtitle{Étude documentaire}

\author{Corentin Smith}
\date{\today}

\begin{document}
\maketitle

\tableofcontents

\newpage
\section{Introduction}

\section{Définition / Histoire}

\subsection{Naissance}
\subsection{Années 90 -- La grande déception}


\section{Technologies}


\subsection{Systèmes de vision}
\subsubsection{Nintendo Virtual Boy}

\subsection{Systèmes audio binauraux}


\subsection{Systèmes maîtres/esclaves}




\subsection{Problématique du temps de latence}

Michael Abrash, \emph{Latency – the sine qua non of AR and VR} (29/12/2012)
\url{http://blogs.valvesoftware.com/abrash/latency-the-sine-qua-non-of-ar-and-vr/}

John Carmack, \emph{Latency Mitigation Strategies} (22/02/2013)
\url{https://web.archive.org/web/20140719053303/http://www.altdev.co/2013/02/22/latency-mitigation-strategies/}


\section{Développements récents et futurs}

\subsection{Oculus}
\subsection{Notion de présence}
\section{Grand public enfin possible ?}

\section{Bibliographie}


\end{document}





